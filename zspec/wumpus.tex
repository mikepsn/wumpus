\documentclass[a4paper]{article}
%\usepackage{a4wide}
\usepackage{oz}
\usepackage[final]{nips_2017}

\title{Wumpus World Formal Specification}
\author{Michael Papasimeon}
\date{}

\begin{document}
\maketitle

\section{Introduction}
This article describes a very (incomplete) formal specification of the Wumpus world game.
The specification is defined in the Z language. The purpose of the
specification is to formally define different types of interactions
between the the environment in which the Wumpus lives in, and the way it
interacts with an agent that is situated in this environment.

\section{Definitions and Utility Functions}
A function $uniform$ is defined that returns a uniform random number
between $x$ and $y$.
\begin{axdef}
	uniform : \nat \cross \nat \tfun \real 
\ST
	\forall x,y : \nat | y \geqslant x 
		\dot x \leqslant uniform(x,y) \leqslant y
\end{axdef}

A function $random$ is defined that returns a random natural number in
the range specified by the parameters to the function.
\begin{axdef}
	random : \natone \cross \natone \tfun \natone
\ST
	\forall x,y : \natone | y \geqslant x 
		\dot random(x,y) \in x..y
\end{axdef}

\section{Agent}
This section defines the details of the Hunter Agent which explores the
cave world in the hope of finding the hidden treasure. The cave
environment in which the Hunter Agent is situated in has a number of
threats, including a number of bottomless pits and a monster called the
wumpus.

The Hunter Agent can only travel from room to room in the cave in
certain directions. He can travel only up or down and left or right. The
agent cannot travel or move diagonally.
\begin{zed}
	DIRECTION ::= Up | Down | Left | Right
\end{zed}

The agent can receive a predefined number of percepts from the
environment, while in any given room of the cave.
\begin{description}
	\item[Stench] If the agent smells a stench, indicates that the wumpus 
	is one of the adjacent rooms (not diagonally).
	\item[Breeze] If the agent feels a breeze, this indicates that there
	is pit in one of the adjacent caves (not diagonally).
	\item[Glitter] If the agent sees glitter, then the agent has found
	the gold in the current room.
	\item[Bump] If the agent perceives a bump that means it has hit the
	wall of the cave and cannot travel further in the direction he was
	travelling in.
	\item[Scream] If the agent hears a scream, this means that the
	wumpus has been killed and any further stench can be ignored.
\end{description}
\begin{zed}
	PERCEPT ::= Stench | Breeze | Glitter | Bump | Scream
\end{zed}

The Hunter Agent only has five different types of actions available to
him. He can either travel one room forward, turn left or right, shoot a
single arrow in the direction the agent is facing, or climb out of the
cave. The agent only has a single arrow available. If he shoots the
arrow it flies in the direction in which the agent was facing until it
hits the wall of the cave or until it hits the wumpus. If it hits the
wumpus, the wumpus is killed. The agent can only climb out of the cave
if there is an opening in the room. In this scenario, the only opening
will be in the room in which the agent entered the cave; that is the
start room.
\begin{zed}
	ACTION ::= Forward | TurnRight | TurnLeft | Shoot | Climb
\end{zed}

The agent schema has a sequence of $PERCEPT$ representing a list of the
percepts the agent is currently perceiving (while in the current
location). Similarly, the schema also has sequence of $ACTION$ to
indicate the actions that it whishes to perform this time step.
The schema indicates that the agent has two basic beliefs about the
world. Firstly, it knows if it is alive, and secondly it knows if it has
shot the arrow it starts out with. Note that the agent doesn't know
anything about it's current position in the cave. In this particular
case the agent isn't storing it's own local coordinates, but it still
doesn't know about the cave's global coordinate system.

\begin{schema}{Agent}
	percepts : \seq PERCEPT \\
	actions : \seq ACTION \\
	alive : \bool \\
	shotArrow : \bool
%\where
\end{schema}

When the agent is initialised it has no percepts and no actions to
perform (and hence the empty sequences). It also knows that it is alive
and that it has yet to shoot the arrow it has.
\begin{schema}{AgentInit}
	\Delta Agent \\
\where
	percepts' = \emptyseq \\
	actions' = \emptyseq \\
	alive' = \true \\
	shotArrow' = \false
\end{schema}

\section{Wumpus}
All the wumpus knows in the wumpus schema is, if it is alive or not.

\begin{schema}{Wumpus}
	alive : \bool
\end{schema}

When the wumpus is initialised it knows it is alive.

\begin{schema}{InitWumpus}
	\Delta Wumpus
\where
	alive' = \true
\end{schema}

\section{Environment}
In this section the environment in which the wumpus and the agent are
situated in is defined. The cave is represented as a two dimensional
grid of rooms. Therefore, and particular room can be defined by its
position in the grid by a tuple representing the grid coordinates.
A $ROOM$ is defined by its position coordinates, which is a relation
between two natural numbers greater than one. Each room can then be
refered to as a row/column tuple pair.
\begin{zed}
	ROOM == \natone \rel \natone
\end{zed}

One particular room is special in the cave. That is the entrance. The
entrance is special because the agent can climb into and climb out of
the cave only when at the entrance. The entrance is defined as the first
row/column pair in the cave $(1,1)$ or $\{1\map1\}$.
\begin{axdef}
	entrance : ROOM
\ST
	entrance = \{ 1 \map 1 \}
\end{axdef}

The cave schema defines the cave firstly by the number of rows and
columns it has. This indirectly defines the number of rooms in the cave.
A finite subset of these rooms will have bottomless pits in them. 
It is not possible for there to be a pit in the entrance room.
The gold will be located in a room in the cave. The wumpus will
also be located in a single room. It is entirely possible that wumpus
and the gold are in the same room. The cave also contains the hunter
agent. The cave schema also keeps track of the agent's location and the
direction it is facing.

\begin{schema}{Cave}
	numRows : \natone \\
	numColumns : \natone \\
	pits : \fset ROOM  \\
	gold : \fsetone ROOM \\
	wumpus : \fsetone ROOM \\
	agentLocation : \fsetone ROOM \\
	agentDirection : DIRECTION \\
	agent : Agent
\where
	\#pits \leqslant numRows * numColumns \\
	\dom pits \subs 1..numRows \\
	\ran pits \subs 1..numColumns \\
	entrance \notin pits \\
	\dom gold \subs 1..numRows \\
	\ran gold \subs 1..numColumns \\
	\#gold = 1 \\
	\#wumpus = 1 \\
	\dom wumpus \subs 1..numRows \\
	\dom wumpus \subs 1..numColumns \\
	\dom agentLocation \subs 1..numRows \\
	\ran agentLocation \subs 1..numColumns \\
\end{schema}

The cave is initialised using three parameters. Firstly the number of
rows and the columns in the cave is set, followed by the probability
that there is a pit in any given room. The cave initialisation schema
also places random pits in the cave, and also randomly places the wumpus
and the gold. It is entirely possible that the gold ends up in a
bottomless pit, in which case it is impossible for the agent to succeed
at its mission. It is also possible for the gold and the wumpus to end
up in the same room.

\begin{schema}{InitCave}
	\Delta Cave \\
	n?, m? : \natone \\
	pitChance? : \real
\where
	numRows' = n? \\
	numColumns' = m? \\
	0.0 \leqslant pitChance? \leqslant 1.0 \\
	\forall i,j : \natone | \\
	\t1	i \in 1..numRows \land j \in 1..numColumns 
		  \land \{ i \map j \} \neq entrance \dot \\
	\t1 uniform(0,1) \leqslant pitChance? \\
	\t1 \implies pits' = pits \union \{ i \map j\} \\
	\forall gx, gy : \natone | \\
	\t1 gx \in 1..numRows \land gy \in 1..numColumns \dot \\
	\t1 gx = random(1,numRows) \land gy = random(1,numColumns) \\
	\t1 \implies gold' = \{ gx \map gy \} \\
	\forall wx, wy : \natone | \\
	\t1 wx \in 1..numRows \land wy \in 1..numColumns \dot \\
	\t1 wx = random(1,numRows) \land wy = random(1,numColumns) \\
	\t1 \implies wumpus' = \{ wx \map wy \} \\
	agentLocation' = entrance \\
	agentDirection' = Left
\end{schema}

\end{document}
